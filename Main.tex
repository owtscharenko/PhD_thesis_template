%% Basierend auf einer TeXnicCenter-Vorlage von Tino Weinkauf.
%%%%%%%%%%%%%%%%%%%%%%%%%%%%%%%%%%%%%%%%%%%%%%%%%%%%%%%%%%%%%%

%%%%%%%%%%%%%%%%%%%%%%%%%%%%%%%%%%%%%%%%%%%%%%%%%%%%%%%%%%%%%
%% HEADER
%%%%%%%%%%%%%%%%%%%%%%%%%%%%%%%%%%%%%%%%%%%%%%%%%%%%%%%%%%%%%
\documentclass[a4paper, twoside, english, 11pt, DIV=12, BCOR=1cm, headsepline, listof=totoc, bibliography=totoc]{scrbook}

%% Deutsche Anpassungen %%%%%%%%%%%%%%%%%%%%%%%%%%%%%%%%%%%%%
\usepackage[british]{babel}
\babelhyphenation[british]{time-stamp}
\usepackage[utf8]{inputenc}    % utf8 support
\usepackage{pgfplots}
\usepackage[]{listings}
\usepackage[T1]{fontenc}
\usepackage[backend=biber, sorting = none, maxcitenames=3, citestyle=numeric-comp, bibencoding=utf8, firstinits=true, isbn=false, block = space]{biblatex} % style=numeric,
\usepackage[babel]{microtype}

\renewcommand*{\multicitedelim}{\addsemicolon\space}
\usepackage[]{owtscharenko-bibstyle}
\usepackage[colorlinks=true, hyperfootnotes=false]{hyperref} % linkcolor = blue
\usepackage[toc, automake, style=long3col, indexonlyfirst, acronym, nomain]{{glossaries-extra}}  % , To have acronyms and glossary
\usepackage{mfirstuc-english}
\glssetcategoryattribute{acronym}{glossdesc}{title}
\MFUhyphenfalse
\MFUnocap{to}
\MFUnocap{of}
\MFUnocap{the}
\MFUnocap{in}
\MFUnocap{on}
\MFUnocap{at}
\MFUnocap{for}
\MFUnocap{and}
\MFUnocap{over}
\MFUnocap{module}
% \Addlcwords{to}
% \Addlcwords{the}
% \Addlcwords{of}
% \renewcommand\titlecap[1]{#1}

% 
%\usepackage{gensymb}}
\usepackage{currvita}
\usepackage{graphicx}
\usepackage{float}
\usepackage{wrapfig}
\usepackage{parskip}
\setlength{\parindent}{0pt}
\usepackage[displaymath, running]{lineno}

\usepackage{etoolbox} %% <- for \cspreto, \csappto

% Patch 'normal' math environments:
\newcommand*\linenomathpatch[1]{%
  \cspreto{#1}{\linenomath}%
  \cspreto{#1*}{\linenomath}%
  \csappto{end#1}{\endlinenomath}%
  \csappto{end#1*}{\endlinenomath}%
}

\linenomathpatch{equation}
\linenomathpatch{gather}
\linenomathpatch{multline}
\linenomathpatch{align}
\linenomathpatch{alignat}
\linenomathpatch{flalign}


%\usepackage{gensymb}
\usepackage[font=normalsize, format=plain, labelfont=bf]{caption}
\usepackage{subcaption}  % for subfigures
\usepackage{setspace} % onehalfspacing
\usepackage{scrlayer-scrpage} % Kopf- und Fusszeilen-Format mit KOMA-Script
\usepackage{lmodern} %Type1-Schriftart for non-english texts
\DeclareFontFamily{OMX}{lmex}{}
\DeclareFontShape{OMX}{lmex}{m}{n}{%
   <->lmex10%  was  <->sfixed*lmex10%
   }{}
\usepackage{isotope} % isotopes printing
\usepackage[toc]{appendix} %,page
\usepackage{multirow}
\usepackage{multicol}  % equation next to each other
\usepackage{pgf} % matlotlib plots with tex links
\usepackage{ctable} % for \specialrule command
\usepackage{footnote}  % footnotes in tables
\usepackage{enumitem}  % no margin in itemize
\usepackage{empheq}  % box around equation
\usepackage[]{todonotes}
\usepackage[nameinlink]{cleveref}
\makesavenoteenv{table}
\makesavenoteenv{tabular}

\setabbreviationstyle[acronym]{long-short}
\loadglsentries{glossary.tex} % file with acronyms
\glssetcategoryattribute{acronym}{glossdesc}{title}
\MFUhyphentrue
\makeglossaries

% counter for footnotes monotonically increasing across the whole document
% \counterwithout{footnote}{chapter}
\renewcommand{\autodot}{}% Remove all end-of-counter dots

%% Header and Footer
\pagestyle{scrheadings}
\clearscrheadfoot
\automark{chapter}
\ihead{\headmark}
\cfoot{\pagemark}

%% Packages for Formula %%%%%%%%%%%%%%%%%%%%%%%%%%%%%%%%%%%%%
\usepackage[binary-units=truem, separate-uncertainty=true]{siunitx} %SI Einheiten
\sisetup{detect-all}
% range-units=single, range-phrase= 
%\sisetup{separate-uncertainty}
\usepackage{bigints} % bigints loads amsmath
% \usepackage{amsmath}
\usepackage{amssymb}
\usepackage{amsthm}
\usepackage{amsfonts}
\usepackage{mathtools}
\usepackage{nicefrac, xfrac}

%% Zeilenabstand %%%%%%%%%%%%%%%%%%%%%%%%%%%%%%%%%%%%%%%%%%%%
\usepackage{setspace}
% \usepackage{times} 
%\singlespacing        %% 1-zeilig (Standard)
\onehalfspacing       %% 1,5-zeilig
% \doublespacing        %% 2-zeilig

%% DRAFT Wasserzeichen %%%%%%%%%%%%%%%%%%%%%%%%%%%%%%%%%%%%%%%%%%%%
% \usepackage{background}
% \backgroundsetup{contents=DRAFT, color=blue!20}
% \linenumbers

%% Own declarations
\DeclareMathOperator\erf{erf}
\DeclareMathOperator\erfi{erfi}

\newcommand{\noi}{\noindent}
\newcommand{\boxedeq}[2]{\begin{empheq}[box={\fboxsep=6pt\fbox}]{align}\label{#1}#2\end{empheq}}
\renewcommand*\descriptionlabel[1]{\hspace\leftmargin$#1$}

\newcommand{\nocontentsline}[3]{}
\newcommand{\tocless}[2]{\bgroup\let\addcontentsline=\nocontentsline#1{#2}\egroup}
\renewcommand*{\acrshort}[1][]{\glsxtrshort[noindex,#1]}

\newlength{\figurewidth}
\newlength{\figureheigth}
\newlength{\tikzwidth}
\newlength{\tikzheigth}
\newlength{\subfigurewidth}
\newlength{\subsubfigurewidth}
\setlength{\figurewidth}{0.8\textwidth}
\setlength{\figureheigth}{0.61804697157\textwidth}  % golden ratio
\setlength{\subfigurewidth}{0.43\textwidth}
\setlength{\subsubfigurewidth}{0.33\textwidth}

% \renewcommand{\textfraction}{0.01}
% \renewcommand{\floatpagefraction}{.1}  % figure only page when more than 10% us used by it
% \renewcommand{\textfraction}{0.2} % minimum amount of page which has to be text
% \renewcommand{\floatpagefraction}{.75}  % figure only page when more than 75% us used by it


% \DeclareSIUnit[]\kursmum{\mu m}
\DeclareSIUnit[]\kursmum{\micro\meter}
\DeclareSIUnit\year{yr}
\DeclareSIUnit\rad{rad}
\DeclareSIUnit\dac{DAC}
\DeclareSIUnit\neq{\mathrm{n}_{\mathrm{eq}}\,\mathrm{cm^{-2}}}
\DeclareSIUnit\permille{\text{\textperthousand}}
\DeclareSIUnit\ehpair{\text{e-h}}
\DeclareSIUnit{\nothing}{\relax}
\DeclareSIUnit\barn{b}
\DeclareSIUnit{\Cmicro}{\SI{100}{\kursmum}}
\DeclareSIUnit\clight{\text{\ensuremath{c}}}
\DeclareSIUnit[per-mode=symbol]\GeVc{\giga\electronvolt\per\clight}
\DeclareSIUnit\csq{\clight\squared}
\DeclareSIUnit[per-mode=symbol]\GeVcsq{\giga\electronvolt\per\csq}
\DeclareSIUnit[]\electron{e^{-}}


\newcommand{\figref}[1]{Figure~\ref{#1}}
\newcommand{\secref}[1]{Section~\ref{#1}}
\newcommand{\chapref}[1]{Chapter~\ref{#1}}
\newcommand{\eqnref}[1]{Equation~\eqref{#1}}
\newcommand{\appref}[1]{Appendix~\ref{#1}}
\newcommand{\tabref}[1]{Table~\ref{#1}}
\newcommand*{\Boltzmann}{k_B}
\Crefname{pluralequation}{Equations}{Equations}

%% Settings %%
\hypersetup{
  colorlinks=True,  % comment for printing
  citecolor=[rgb]{0.282623, 0.140926, 0.457517},  % viridis darkslateblue
  linkcolor=[rgb]{0.282623, 0.140926, 0.457517},  % viridis darkslateblue
  urlcolor=[rgb]{0.282623, 0.140926, 0.457517}  % viridis darkslateblue
  }
  % \hypersetup{
  % colorlinks=True,  % comment for printing
  % citecolor=[rgb]{0.0, 0.0, 0.0},  % viridis darkslateblue
  % linkcolor=[rgb]{0.0, 0.0, 0.0},  % viridis darkslateblue
  % urlcolor=[rgb]{0.0, 0.0, 0.0}  % viridis darkslateblue
  % }
\addbibresource{PhD_thesis.bib}
\graphicspath{{./figures/}}


\definecolor{codegreen}{rgb}{0,0.6,0}
\definecolor{codegray}{rgb}{0.5,0.5,0.5}
\definecolor{codepurple}{rgb}{0.58,0,0.82}
\definecolor{backcolour}{rgb}{0.95,0.95,0.92}

\lstdefinestyle{mystyle}{
    % backgroundcolor=\color{backcolour},   
    commentstyle=\color{codegreen},
    keywordstyle=\color{magenta},
    numberstyle=\tiny\color{codegray},
    stringstyle=\color{codepurple},
    basicstyle=\ttfamily\footnotesize,
    breakatwhitespace=false,         
    breaklines=true,                 
    captionpos=b,                    
    keepspaces=true,                 
    numbers=left,                    
    numbersep=5pt,                  
    showspaces=false,                
    showstringspaces=false,
    showtabs=false,                  
    tabsize=4
}
\widowpenalty=10000
\clubpenalty=10000
\lstset{style=mystyle}

%\input{Glossary.tex}
\hypersetup{breaklinks=true}
\sisetup{range-phrase=%
 \ifmmode\mathbin{-}
 \else
  \thinspace\textendash\thinspace
 \fi%
}


%%%%%%%%%%%%%%%%%%%%%%%%%%%%%%%%%%%%%%%%%%%%%%%%%%%%%%%%%%%%%
%% DOKUMENT
%%%%%%%%%%%%%%%%%%%%%%%%%%%%%%%%%%%%%%%%%%%%%%%%%%%%%%%%%%%%%
\newcommand*{\handInInstitute}{Eingereicht bei der Naturwissenschaftlich-Technischen Fakultät \\ der Universität Siegen }
\author{vorgelegt von \\ Nikolaus G. G. Owtscharenko\\
aus \textit{some city} \vspace{30mm} \\ \handInInstitute}

\title{Your thesis title\\}
\subtitle{\vspace{20mm} DISSERTATION \\
	zur Erlangung des Grades eines Doktors \\
	der Naturwissenschaften \\
	\vspace{15mm}
	%\includegraphics*[scale=0.57]{Graphics/Cover_tiny.jpg}	%  kleines logo
}
\date{2023}

\newcommand{\thesisabstract}{
The abstract in english, should not exceed one DIN-A4 page.
}

\newcommand{\thesisabstractGerman}{
Das gleiche gilt für die deutsche Kurzzusammenfassung.
}

\begin{document}
\frontmatter
\renewcommand{\glsnamefont}[1]{\textbf{#1}}
%\include{./chapter/PhD_Cover}
\maketitle  % COVER, only front side of title page

\lowertitleback{
  Betreuer und erster Gutachter: Prof. John Doe\\
  Zweiter Gutachter: Prof. Jane Doe\\
  Tag der mündlichen Prüfung: XX.XX.XXXX}
\maketitle % second title including backside
\section*{Abstract}
\thesisabstract
\newpage
\section*{Zusammenfassung}
\thesisabstractGerman


\let\backupskip\chapterheadstartvskip
\renewcommand*\chapterheadstartvskip{\vspace*{-\topskip}}
\tableofcontents
\let\chapterheadstartvskip\backupskip

\newpage
\mainmatter
% \setcounter{page}{1}  % reset page counters
\chapter{Introduction}
\label{ch:intro}
You should introduce the topic of the thesis and what the reader can expect to learn.
\chapter{This is a chapter}
\label{ch:example}
Some introductory text describing the chapter contents. Also, an example for citing a range of pages in a reference \cite[p.~2~ff]{AguilarBenitez1988}.\\
The package \texttt{siunitx} provides an easy and foolproof solution for giving values with units and their uncertainties: $v_x = \pm\qty[per-mode=symbol]{2.610(5)}{\centi\meter\per\second}$. One can even define their own units as for example $\unit{Neq}$ or $\unit{GeVc}$.\\
There is also a glossary. The \texttt{glossaries-extra} package will take care of formatting and keep score of whether and where the first mention of any abbreviation is placed. Example for the  "SPS": First mention \gls{sps}, second mention \gls{sps}. The abbreviations are defined in "glossary.tex" in the root folder.\\
The project uses biber for bibliography management and the compilation procedure is:
\begin{enumerate}
    \item pdflatex,
    \item bibtex,
    \item "makeglossaries",
    \item pdflatex,
    \item pdflatex.
\end{enumerate}

I recommend the VS code editor\footnote{you can find it here: \href{https://code.visualstudio.com}{https://code.visualstudio.com}} with "LaTeX workshop" and "LTeX" extensions for writing. The VS code settings for the project are provided in the repository.\\
I used jabref (\href{https://www.jabref.org}{https://www.jabref.org}) for reference management. It is possible to get a .bib file as output which can easily be linked in the latex project.

\section{This is a section}
\label{sec:example}

\begin{table}
    \renewcommand{\arraystretch}{1.8}
    \begin{center}
    % \begin{tabular}{>{\raggedright}p{2cm} >{\raggedright}p{2cm} >{\raggedright}p{1cm} >{\raggedright}p{1cm} >{\raggedright}p{1cm} >{\raggedright\arraybackslash}p{1cm}}
    \begin{tabular}{ccccc >{\centering\arraybackslash}p{2.5cm}}
    &\bfseries runs &\bfseries passive material &\bfseries ECC &\bfseries films &\bfseries integrated p.o.t. $\times 10^5 $\\
    \hline
    \bfseries CHARM 1 & 6 & -- & \qty[]{28}{\milli\meter} Pb/W + \num{29} films & 174 & 5.4 \\
    \bfseries CHARM 2 & 6 & \qty[]{28}{\milli\meter} Pb & \qty[]{28}{\milli\meter} Pb + \num{29} films & 174 & 5.2 \\
    \bfseries CHARM 3 & 3 & \qty[]{56}{\milli\meter} Pb & \qty[]{56}{\milli\meter} Pb + \num{57} films & 174 & 1.0 \\
    \bfseries CHARM 4 & 3 & \qty[]{113}{\milli\meter} Pb & \qty[]{56}{\milli\meter} Pb + \num{57} films & 171 & 0.8 \\
    \bfseries CHARM 5 & 3 & \qty[]{168}{\milli\meter} Pb & \qty[]{56}{\milli\meter} Pb + \num{57} films & 171 & 1.6 \\
    \bfseries CHARM 6 & 3 & \qty[]{224}{\milli\meter} Pb & \qty[]{56}{\milli\meter} Pb + \num{57} films & 171 & 1.6 \\
    \hline
    \bfseries Total & 24 & & & 1032 & 15.6
    \end{tabular}
    \end{center}
    \caption{An exemplary table.}
    \label{tab:example}
\end{table}

\begin{figure}[htbp]
    \centering
    \includegraphics[width=0.49\textwidth]{"sm_left_handed"}
    \includegraphics[width=0.49\textwidth]{"nuMSM.pdf"}
    \caption{Example for two figure graphics. Figures from \cite[2]{Gninenko2013}}
    \label{fig:example_two_fig}
\end{figure}

\section{Itemizing}
\label{sec:itemizing_example}
It is possible to make a list in the itemize environment without bullets: \Cref{subsec:track_fitting}:
\begin{itemize}
    \item[]\textbf{First item:}
    \item[]\textbf{Second item:}
\end{itemize}

This is what an equation with label looks like:
\begin{equation}
    \sigma_\mathrm{proj}^2 = z^2 \cdot \sigma_\mathrm{slope}^2 + \sigma_\mathrm{pos}^2 + 2 z \cdot \sigma_\mathrm{slope/pos}
    \label{eq:track_projection_sigma}
\end{equation}

With the \texttt{cleveref} package one can reference any label and the specific descriptor is automatically added, e.g. "Equation" for \Cref{eq:track_projection_sigma} and "Figure" for \Cref{fig:example_two_fig}.
Matrices can be written like in the following \Cref{eg:matrix_example}.

Where
\begin{align}
    \boldsymbol{\alpha} &= \begin{pmatrix}
                    x \\
                    y \\
                    \theta_{xz}\\
                    \theta_{yz}\\
                   \end{pmatrix}
\label{eg:matrix_example}
\end{align}
\chapter{Summary}
\label{ch:summary}

A summary of what was found and why it was investigated.
%\setcounter{tocdepth}{0}  % do not show chapter in table of contents
\begin{appendices}
\renewcommand{\thefigure}{A.\arabic{figure}}
\renewcommand{\thetable}{A.\arabic{table}}
\markboth{Appendix}{Appendix}
\chapter*{Appendix}
\label{ch:appendix}

\section*{Some appendix section}
\label{app:example}
Following is an example of how to provide code:
\begin{minipage}{\textwidth}
\begin{lstlisting}[caption={Event data format as accepted by ControlHost. The number of RawDataHits is not limited, the number of hits which are read is determined by the size parameter in the header.}, label={lst:dataformat}, language=C++]
struct DataFrameHeader
{
uint16_t size;	// Length of the data frame in bytes (including header).
uint16_t partitionId;	// Identifier of the subdetector and partition.
uint32_t cycleIdentifier;	// SHiP cycle identifier as received from TFC.
uint32_t frameTime;	// Frame time in 25ns clock periods
uint16_t timeExtent;	// sequential trigger number 
uint16_t flags;	// Version, truncated, etc.
};

struct RawDataHit
{
uint16_t channelId;	// Channel Identifier
uint16_t hitTime;	// Hit time, coarse 25ns based time in MSByte, fine time in LSByte
uint16_t extraData[0];	// Optional subdetector specific data items
};
\end{lstlisting}
\end{minipage}


\end{appendices}
\setlength\LTleft{-5pt}
\printglossary[title={Abbreviations}]
\printbibliography
\markboth{}{}
\chapter*{Acknowledgements}
\label{ch:acknowledge}

Here is the place to thank people supporting you during your work.
\setlength{\cvlabelwidth}{40mm}
\chapter*{}
\vspace{-30mm}
\begin{cv}{Curriculum Vitae}
\vspace{5mm}
\begin{cvlist}{Personal Details}
    \item [Name] Your full name
    \item [Place of birth] some city
    \item [Date of birth] XX.XX.XXXX
\end{cvlist}
\vspace{0.5mm}
\begin{cvlist}{Education}
    \item [Jan.~2000 -- June~2008] \textbf{A-Level} (A-level equivalent), some school, some city
                        Final grade: 1.0
    \item [Oct.~2000 -- Sept.~2000] \textbf{BSc in Physics}, some university, some city
                        Final grade : 1.0
    \item [Aug.~2000]          \textbf{Bachelor Thesis}\\
                        Thesis Topic: \textit{Your thesis topic}\\
                        Group of your professor
                        Grade : 1.0
    \item [May~2000 -- Aug.~2000] \textbf{MSc in Physics}, some university, some city\\
                        Focus: your specialization 
                        Final grade : 1.0
    \item [July~2000]          \textbf{Master Thesis}\\
                        Thesis Topic: \textit{Your thesis topic}\\
                        Group of your professor
                        Grade : 1.0
    \item [Jan.~2000 -- Dec.~2000] \textbf{PhD student}, some university, some city \\
                        Group of your professor

\end{cvlist}
\date{}
\end{cv}


\end{document}
