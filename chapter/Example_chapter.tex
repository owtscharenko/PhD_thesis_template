\chapter{This is a chapter}
\label{ch:example}
Some introductory text describing the chapter contents. Also, an example for citing a range of pages in a reference \cite[p.~2~ff]{AguilarBenitez1988}.\\
The package \texttt{siunitx} provides an easy and foolproof solution for giving values with units and their uncertainties: $v_x = \pm\qty[per-mode=symbol]{2.610(5)}{\centi\meter\per\second}$. One can even define their own units as for example $\unit{Neq}$ or $\unit{GeVc}$.\\
There is also a glossary. The \texttt{glossaries-extra} package will take care of formatting and keep score of whether and where the first mention of any abbreviation is placed. Example for the  "SPS": First mention \gls{sps}, second mention \gls{sps}. The abbreviations are defined in "glossary.tex" in the root folder.\\
The project uses biber for bibliography management and the compilation procedure is:
\begin{enumerate}
    \item pdflatex,
    \item bibtex,
    \item "makeglossaries",
    \item pdflatex,
    \item pdflatex.
\end{enumerate}

I recommend the VS code editor\footnote{you can find it here: \href{https://code.visualstudio.com}{https://code.visualstudio.com}} with "LaTeX workshop" and "LTeX" extensions for writing. The VS code settings for the project are provided in the repository.\\
I used jabref (\href{https://www.jabref.org}{https://www.jabref.org}) for reference management. It is possible to get a .bib file as output which can easily be linked in the latex project.

\section{This is a section}
\label{sec:example}

\begin{table}
    \renewcommand{\arraystretch}{1.8}
    \begin{center}
    % \begin{tabular}{>{\raggedright}p{2cm} >{\raggedright}p{2cm} >{\raggedright}p{1cm} >{\raggedright}p{1cm} >{\raggedright}p{1cm} >{\raggedright\arraybackslash}p{1cm}}
    \begin{tabular}{ccccc >{\centering\arraybackslash}p{2.5cm}}
    &\bfseries runs &\bfseries passive material &\bfseries ECC &\bfseries films &\bfseries integrated p.o.t. $\times 10^5 $\\
    \hline
    \bfseries CHARM 1 & 6 & -- & \qty[]{28}{\milli\meter} Pb/W + \num{29} films & 174 & 5.4 \\
    \bfseries CHARM 2 & 6 & \qty[]{28}{\milli\meter} Pb & \qty[]{28}{\milli\meter} Pb + \num{29} films & 174 & 5.2 \\
    \bfseries CHARM 3 & 3 & \qty[]{56}{\milli\meter} Pb & \qty[]{56}{\milli\meter} Pb + \num{57} films & 174 & 1.0 \\
    \bfseries CHARM 4 & 3 & \qty[]{113}{\milli\meter} Pb & \qty[]{56}{\milli\meter} Pb + \num{57} films & 171 & 0.8 \\
    \bfseries CHARM 5 & 3 & \qty[]{168}{\milli\meter} Pb & \qty[]{56}{\milli\meter} Pb + \num{57} films & 171 & 1.6 \\
    \bfseries CHARM 6 & 3 & \qty[]{224}{\milli\meter} Pb & \qty[]{56}{\milli\meter} Pb + \num{57} films & 171 & 1.6 \\
    \hline
    \bfseries Total & 24 & & & 1032 & 15.6
    \end{tabular}
    \end{center}
    \caption{An exemplary table.}
    \label{tab:example}
\end{table}

\begin{figure}[htbp]
    \centering
    \includegraphics[width=0.49\textwidth]{"sm_left_handed"}
    \includegraphics[width=0.49\textwidth]{"nuMSM.pdf"}
    \caption{Example for two figure graphics. Figures from \cite[2]{Gninenko2013}}
    \label{fig:example_two_fig}
\end{figure}

\section{Itemizing}
\label{sec:itemizing_example}
It is possible to make a list in the itemize environment without bullets: \Cref{subsec:track_fitting}:
\begin{itemize}
    \item[]\textbf{First item:}
    \item[]\textbf{Second item:}
\end{itemize}

This is what an equation with label looks like:
\begin{equation}
    \sigma_\mathrm{proj}^2 = z^2 \cdot \sigma_\mathrm{slope}^2 + \sigma_\mathrm{pos}^2 + 2 z \cdot \sigma_\mathrm{slope/pos}
    \label{eq:track_projection_sigma}
\end{equation}

With the \texttt{cleveref} package one can reference any label and the specific descriptor is automatically added, e.g. "Equation" for \Cref{eq:track_projection_sigma} and "Figure" for \Cref{fig:example_two_fig}.
Matrices can be written like in the following \Cref{eg:matrix_example}.

Where
\begin{align}
    \boldsymbol{\alpha} &= \begin{pmatrix}
                    x \\
                    y \\
                    \theta_{xz}\\
                    \theta_{yz}\\
                   \end{pmatrix}
\label{eg:matrix_example}
\end{align}